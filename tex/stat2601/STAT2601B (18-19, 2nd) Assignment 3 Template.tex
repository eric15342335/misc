\documentclass[a4paper,11pt]{article}
\usepackage[margin=1in]{geometry}
\usepackage{amsmath}
\usepackage{amssymb}

%-----------------------------------------------------------
% You do not need to change anything here
\newcommand{\info}[4]{
  \begin{center}
    {\bfseries \Large STAT2601B Probablity and Statistics I}
    {\bfseries (2018-2019 Second Semester)} \\
    \vspace*{3ex}
    {\bfseries \Large \underline{Assignment 3}}
  \end{center}
  Due Date: 11/4/2019 (17:00)
  \begin{flushleft}
  \begin{tabular}{@{}ll@{}}
    Name & #1 \\
    Student No. & #2 \\
    Class No. & #3 \\
    Tutorial Group & #4
  \end{tabular}
  \end{flushleft}
}
\renewcommand\labelenumiii{(\roman{enumiii})}
\renewcommand\theenumiii\labelenumiii
%-----------------------------------------------------------

\begin{document}
  \info{Your Name}
       {Your Student No.}
       {Your Class No.}
       {Your Tutorial Group}
  \begin{enumerate}
    % Question 1
    \item \textit{(6 points)}
    % Question 2
    \item \textit{(6 points)}
    % Question 3
    \item \textit{(10 points)}
    \begin{enumerate}
      % Question 3(a)
      \item
      % Question 3(b)
      \item
      % Question 3(c)
      \item
    \end{enumerate}
    % Question 4
    \item \textit{(10 points)}
    \begin{enumerate}
      % Question 4(a)
      \item
      % Question 4(b)
      \item
      % Question 4(c)
      \item
    \end{enumerate}
    % Question 5
    \item \textit{(16 points)}
    \begin{enumerate}
      % Question 5(a)
      \item
      \begin{enumerate}
        % Question 5(a)(i)
        \item
        % Question 5(a)(ii)
        \item
        % Question 5(a)(iii)
        \item
      \end{enumerate}
      % Question 5(b)
      \item
      \begin{enumerate}
        % Question 5(b)(i)
        \item
        % Question 5(b)(ii)
        \item
        % Question 5(b)(iii)
        \item
      \end{enumerate}
    \end{enumerate}
    % Question 6
    \item \textit{(10 points)}
    \begin{enumerate}
      % Question 6(a)
      \item
      % Question 6(b)
      \item
      % Question 6(c)
      \item
      % Question 6(d)
      \item
    \end{enumerate}
    % Question 7
    \item \textit{(18 points)}
    \begin{enumerate}
      % Question 7(a)
      \item
      % Question 7(b)
      \item
      % Question 7(c)
      \item
      \begin{enumerate}
        % Question 7(c)(i)
        \item
        % Question 7(c)(ii)
        \item
        % Question 7(c)(iii)
        \item
        % Question 7(c)(iv)
        \item
      \end{enumerate}
      % Question 7(d)
      \item
      % Question 7(e)
      \item
    \end{enumerate}
    % Question 8
    \item \textit{(16 points)}
    \begin{enumerate}
      % Question 8(a)
      \item
      % Question 8(b)
      \item
      % Question 8(c)
      \item
      % Question 8(d)
      \item
      % Question 8(e)
      \item
      % Question 8(f)
      \item
    \end{enumerate}
    % Question 9
    \item \textit{(12 points)}
    \begin{enumerate}
      % Question 9(a)
      \item
      % Question 9(b)
      \item
      % Question 9(c)
      \item
      % Question 9(d)
      \item
      % Question 9(e)
      \item
      % Question 9(f)
      \item
    \end{enumerate}
    % Question 10
    \item \textit{(16 points)}
    \begin{enumerate}
      % Question 10(a)
      \item
      \begin{enumerate}
        % Question 10(a)(i)
        \item
        % Question 10(a)(ii)
        \item
        % Question 10(a)(iii)
        \item
      \end{enumerate}
      % Question 10(b)
      \item
      \begin{enumerate}
        % Question 10(b)(i)
        \item
        % Question 10(b)(ii)
        \item
        % Question 10(b)(iii)
        \item
      \end{enumerate}
    \end{enumerate}
  \end{enumerate}
\end{document}
